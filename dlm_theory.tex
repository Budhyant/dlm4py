\documentclass[12pt]{article}

\usepackage[top=1in, bottom=1in, left=1in, right=1in]{geometry}
\usepackage{graphicx}
\usepackage{epsfig}
\usepackage{amsmath,amssymb,amsthm}
\usepackage{booktabs}
\usepackage{mathptmx}
\usepackage{caption}
\usepackage{subcaption}

\newtheorem{theorem}{Theorem}[section]
\newtheorem{define}{Definition}[section]
\newtheorem{lemma}{Lemma}[section]
\newtheorem{corollary}{Corollary}[section]
\newtheorem{proposition}{Proposition}[section]

\usepackage[colorlinks, citecolor=black, linkcolor=black, %
filecolor=black, urlcolor=blue]{hyperref}
\usepackage[numbers]{natbib}
\usepackage{doi}
\usepackage{algpseudocode}

\renewcommand{\floatpagefraction}{0.9}

\title{Doublet Lattice Theory Document}

\author{Graeme J. Kennedy
  \thanks{Assistant Professor, School of Aerospace Engineering,
    Georgia Institute of Technology, Atlanta, GA, email:
    graeme.kennedy@ae.gatech.edu, phone: 404-894-8911}} 
\date{}

% Commands for partial derivatives and fractions
\newcommand{\p}{\partial}
\newcommand{\f}{\frac}
\newcommand{\mb}{\mathbf}
\newcommand{\mbs}{\boldsymbol}
\newcommand{\ds}{\displaystyle}

\begin{document}

\maketitle

\section{Introduction}

This document contains a brief outline of the DLM theory as it is
implemented within the python/Fortran DLM contained within this
repository. This implementation is based in part on the method of
Albano and Rodden, 1969, and the extension by Rodden, Taylor and
McIntosh, 1998.

The pressure potential equation can be written as follows:
%
\begin{equation*}
  \beta^{2} p_{xx} + p_{yy} + p_{zz} - \f{2U}{a^2} p_{xt} - \f{1}{a^2} p_{tt} = 0
\end{equation*}
where $p$ is both the pressure and the pressure potential. Here the
free-stream is aligned with the $x$-coordinate axis. Solutions to the
pressure potential equation can be written as follows:
%
\begin{equation*}
  p = \f{1}{R} f(t - \tau)
\end{equation*}
where $f$ is a general function and $\tau$ and $R$ are defined as
follows:
%
\begin{equation*}
  \begin{aligned}
    R & = \sqrt{(x - \xi)^2 + \beta^2((y - \eta)^2 + (z - \zeta)^2)} \\
    \tau & = \f{R - M(x - \xi)}{a\beta^2}
  \end{aligned}
\end{equation*}
Now, specializing this general expression to harmonic behavior
in time, we can arrive at the following expression:
%
\begin{equation*}
  p = \f{1}{R} e^{\left[ \f{i\omega}{a\beta^2}(M(x - \xi) - R)\right]} e^{i\omega t}
\end{equation*}


\begin{equation*}
  \begin{aligned}
    \psi & = \f{\p}{\p z} \left[ \f{1}{R} e^{\left[ \f{i\omega}{a\beta^2}(M(x - \xi) - R)\right]} \right] \\
    & = \beta^2 (z - \zeta)\left[ \f{1}{R^2} - \f{1}{R^3}\right]e^{\left[ \f{i\omega}{a\beta^2}(M(x - \xi) - R)\right]}
  \end{aligned}
\end{equation*}


\begin{equation*}
  \phi = -\f{1}{U} e^{-\f{i\omega x}{U}} \int_{-\infty}^{x} e^{\f{i\omega x}{U}} \psi \, dx
\end{equation*}

\begin{equation*}
  \hat{w} = - \f{1}{U} \f{\p \phi}{\p z} = \f{1}{U^2} e^{-\f{i\omega x}{U}} \int_{-\infty}^{x} e^{\f{i\omega x}{U}} \psi \, dx
\end{equation*}


\begin{equation*}
  \hat{w} = \f{1}{4\pi \rho U^2} \int_{S} \Delta p e^{-\f{i\omega(x - \xi)}{U}} 
  \f{\p^2}{\p z^2} \left[ 
    \int_{-\infty}^{x - \xi} \f{1}{R} e^{i\omega\left( \f{x}{U} - \f{R - Mx}{a\beta^2} \right)} \, dx 
    \right] \, dS
\end{equation*}


The doublet lattice method (DLM) is based on the following expression
for the normal wash, $w(x, s)$ on a wing in an oscillating flow where
the $x$-direction is parallel to the free-stream direction:
%
\begin{equation}
  \label{eqn:normal-wash}
  \hat{w}(x, s) = \f{1}{8\pi} 
  \int_{S} \Delta C_p(\xi, \eta) \, K(x - \xi, s - \eta) \, d\xi d\eta 
\end{equation}
where the kernel function $K$ is given by the following expression:
%
\begin{equation*}
  K = e^{-i\omega x_{0}/U} \left[ \f{K_1 T_1}{r_{1}^2} + \f{K_2 T_2}{r_{1}^4} \right]
\end{equation*}
where:
%
\begin{equation*}
  x_{0} = x - \xi \qquad
  y_{0} = y - \eta \qquad
  z_{0} = z - \zeta \qquad 
  r_{1} = \sqrt{y_{0}^2 + z_{0}^2}
\end{equation*}

The terms $T_{1}$ and $T_{2}$ are the normal and transverse
contributions to the normal wash defined as follows:
\begin{equation*}
  \begin{aligned}
    T_{1} & = \cos(\gamma_r - \gamma_s) \\
    T_{2} & = (z_{0} \cos \gamma_r - y_{0} \sin \gamma_r)(z_{0} \cos \gamma_r - y_{0} \sin \gamma_r)
  \end{aligned}
\end{equation*}

The terms $K_{1}$ and $K_{2}$ are contributions to the kernel function
given as follows:
%
\begin{equation*}
  \begin{aligned}
    K_{1} & = I_{1} + \f{Mr_{1}}{R} \f{e^{-ik_1 u_1}}{\sqrt{1 + u_{1}^2}} \\
    K_{2} & = -3I_{2} - i \f{k_1 M^2r_1^2}{R^2} \f{e^{-ik_1 u_1}}{\sqrt{1 + u_{1}^2}} 
    - \f{Mr_{1}}{R} \left[ (1 + u_{1}^2) \f{\beta^2r_{1}^2}{R^2} + 
      2 + \f{Mr_1u_{1}}{R} \right]\f{e^{-ik_1 u_1}}{(1 + u_{1}^2)^{3/2}} \\
  \end{aligned}
\end{equation*}
where the following definitions are used:
\begin{equation*}
  \beta^2 = 1 - M^2 \qquad 
  k_{1} = \f{\omega r_{1}}{U} \qquad 
  R^2 = x_{0}^2 + \beta^2r_{1}^2 \qquad 
  u_{1} = \f{MR - x_{0}}{\beta^2r_1}
\end{equation*}

Finally, $I_{1}$ and $I_{2}$ are integrals that are given as follows:
%
\begin{equation}
  \label{eqn:I-integrals}
  \begin{aligned}
    I_{1} = \int_{u_{1}}^{\infty} \f{e^{-ik_1u}}{(1 + u)^{3/2}} \, du \\
    I_{2} = \int_{u_{1}}^{\infty} \f{e^{-ik_1u}}{(1 + u)^{5/2}} \, du 
  \end{aligned}
\end{equation}

The doublet lattice method contains two essential components:
\begin{enumerate}
\item The integral expressions~\eqref{eqn:I-integrals} are
  approximated using an approximation of the function
  $1 - u/\sqrt{1 + u^2}$
\item The kernel function itself is approximated using a quartic
  expression and integrated across the panel length
\end{enumerate}

\section{Approximate $I_{1}$ and $I_{2}$ integrals}

The $I_{1}$ and $I_{2}$ integrals can be approximated in a similar
manner. In both cases, the integrals can be expressed as an integral
in $1 - u/\sqrt{1 + u^2}$. This can be obtained by integrating by
parts and shifting the expression for the integrand. For $I_{1}$, 
we can obtain the following expression:
\begin{equation*}
  \begin{aligned}
    I_{1} & = \int_{u_{1}}^{\infty} \f{e^{-ik_1u}}{(1 + u)^{3/2}} \, du \\
    %
    & = \left. \f{u e^{-ik_1 u}}{\sqrt{1 + u^2}} \right|_{u=u_{1}}^{\infty} +
    ik_{1} \int_{u_{1}}^{\infty} \f{ue^{-ik_1 u}}{\sqrt{1 + u^2}} \, du \\
    %
    & = \left[1 - \f{u_{1}}{\sqrt{1 + u_1^2}} \right]e^{-ik_{1}u_{1}} -
    ik_{1} \int_{u_{1}}^{\infty} \left[1 - \f{u}{\sqrt{1 + u^2}} \right] e^{-ik_1 u} \, du  \\
    %
    & = \left[1 - \f{u_{1}}{\sqrt{1 + u_1^2}} \right]e^{-ik_{1}u_{1}} - ik_{1} I_{0}
  \end{aligned}
\end{equation*}
where the integral $I_{0}$ is defined as:
\begin{equation*}
  I_{0} = \int_{u_{1}}^{\infty} \left[ 1 - \f{u}{\sqrt{1 + u^2}} \right] e^{-ik_1 u} \, du
\end{equation*}

The second integral can be expressed as follows:
\begin{equation*}
    I_{2} = \int_{u_{1}}^{\infty} \f{e^{-ik_1u}}{(1 + u^2)^{5/2}} \, du 
\end{equation*}

This integral can be simplified as follows:
%
\begin{equation*}
  3 I_{2} = \left[ (2 + ik_{1}u_{1}) \left( 1 - \f{u_{1}}{\sqrt{1 + u_{1}^2}}\right) 
    - \f{u_{1}}{(1 + u_{1}^2)^{3/2}}\right]e^{-ik_{1}u_{1}} - ik_{1} I_{0} + k_{1}^2J_{0}
\end{equation*}

The last integral $J_{0}$ is defined as follows:
%
\begin{equation*}
  J_{0} = \int_{u_{1}}^{\infty} u \left[ 1 - \f{u}{\sqrt{1 + u^2}} \right] \, du
\end{equation*}

The integrals $I_{0}$ and $J_{0}$ can be obtained based on the
following approximation:
%
\begin{equation*}
  1 - \f{u}{\sqrt{1 + u^{2}}} = \sum_{j=1}^{n} a_{j} e^{-p_{j}u}
\end{equation*}
%
where $n = 12$, $p_{j} = b\,2^{j}$, $b = 0.009054814793$, and the
remaining coefficients are given in Table~\ref{table:a-coef}.
%
\begin{table}[h]
  \begin{center}
    \begin{tabular}{l}
      \toprule
      Coefficients: $a_{j}$ \\
      \midrule
      0.000319759140 \\
      -0.000055461471 \\
      0.002726074362 \\
      0.005749551566 \\
      0.031455895072 \\
      0.106031126212 \\
      0.406838011567 \\
      0.798112357155 \\
      -0.417749229098 \\
      0.077480713894 \\
      -0.012677284771 \\
      0.001787032960 \\
      \bottomrule
    \end{tabular}
  \end{center}
  \caption{Coefficients for the 12-term approximation}
  \label{table:a-coef}
\end{table}

Based on these approximations, the integral $I_{0}$ can be
approximated as follows:
%
\begin{equation*}
  I_{0} \approx \int_{u_{1}}^{\infty} \sum_{j=1}^{n} a_{j} e^{-p_{j}u - ik_{1}u} \, du 
  %
  = \sum_{j=1}^{n} \f{a_{j}e^{-(p_{j} + i k_{1})u_{1}}}{p_{j}^2 + k_{1}^2} (p_{j} - i k_1)
\end{equation*}
%
The integral $J_{0}$ can be approximated in a similar manner as
follows:
%
\begin{equation*}
  J_{0} \approx \int_{u_{1}}^{\infty} \sum_{j=1}^{n} a_{j} u e^{-(p_{j} + ik_{1})u} \, du 
  %
  = \sum_{j=1}^{n} \f{a_{j}e^{-(p_{j} + i k_{1})u_{1}}((p_{j} + i k_1)u_{1} + 1)(p_{j} - ik_{1})^2}{(p_{j}^2 + k_{1}^2)^2}
\end{equation*}

\section{Integration of the approximate kernel function}



\section{Steady-state horseshoe vortex code}

The steady state components are calculated based on a horseshoe vortex
formulation. In this approach, all that is required is a
Pradtl-weighted distance the bound vortex start and end locations to
the receiving point. 

These distances are denoted as follows:
%
\begin{equation*}
  \mb{a} = \begin{bmatrix} 
    (x_{r} - x_{i})/\beta \\
    y_r - y_i \\
    z_r - z_i \\
  \end{bmatrix}
  \qquad
  \mb{b} = \begin{bmatrix} 
    (x_{r} - x_{o})/\beta \\
    y_r - y_o \\
    z_r - z_o \\
  \end{bmatrix}
\end{equation*}
%
where $(x_r, y_r, z_r)$ is the receiving point, $(x_{i}, y_{i},
z_{i})$ is the inboard vortex line point and $(x_{o}, y_{o}, z_{o})$
is the outboard vortex point. Note that $\beta = \sqrt{1 - M^2}$.

The velocity induced by the bound vortex segment is given as follows:
%
\begin{equation*}
  \mb{v}_{b} = \f{\Gamma}{4\pi} \f{\mb{a} \times \mb{b} (A + B)}{AB(AB + \mb{a}\cdot \mb{b})}
\end{equation*}
where $A = \sqrt{\mb{a}\cdot \mb{a}}$ and $B =
\sqrt{\mb{b}\cdot\mb{b}}$.  Note that this formulation for $\mb{v}$ is
singular only when $\mb{a}$ and $\mb{b}$ are co-linear and lie
pointing towards one another such that $AB = - \mb{a}\cdot \mb{b}$.
Based on the definitions of $\mb{a}$ and $\mb{b}$, this singularity
will only occur when the receiving point lies between the inboard and
outboard points.

To account for the inboard and outboard vorticies traveling from a
point projected infinitely far downstream, we define the vectors
$\mb{c}$ and $\mb{d}$ as follows:
%
\begin{equation*}
  \mb{c} = \begin{bmatrix}
    -\infty \\
    y_{r} - y_{i} \\ 
    z_{r} - z_{i}
  \end{bmatrix}
  \qquad
  \mb{d} = \begin{bmatrix}
    -\infty \\
    y_{r} - y_{o} \\ 
    z_{r} - z_{o}
  \end{bmatrix}
\end{equation*}

This yields the following expression for the contribution from the
inboard vortex segment:
%
\begin{equation*}
  \begin{aligned}
    \mb{v}_{i} & = \lim_{C \rightarrow \infty} \f{\Gamma}{4\pi} \f{\mb{c} \times \mb{a}(A + C)}{AC(AC + \mb{c}\cdot \mb{a})} \\
    & = \f{\Gamma}{4\pi} \f{-\mb{i} \times \mb{a}}{A(A - \mb{i}\cdot \mb{a})} \\
    & = \f{\Gamma}{4\pi} \f{\mb{a} \times \mb{i}}{A(A - \mb{i}\cdot \mb{a})} \\
    & = \f{\Gamma}{4\pi} \f{a_{z} \mb{j} - a_{y}\mb{k}}{A(A - a_x)}
  \end{aligned}
\end{equation*}
Note that we have used the fact that $\lim_{C \rightarrow \infty}
\mb{c}/C = -\mb{i}$.  Now, accounting for the remaining vortex from
the outboard point yields the following:
\begin{equation*}
  \begin{aligned}
    \mb{v}_{o} & = \lim_{D \rightarrow \infty}\f{\Gamma}{4\pi} \f{ \mb{b} \times \mb{d}(D + B)}{BD(BD + \mb{b}\cdot \mb{d})} \\
    & = \f{\Gamma}{4\pi} \f{-\mb{b} \times \mb{i}}{B(B - \mb{i}\cdot \mb{b})} \\
    & = \f{\Gamma}{4\pi} \f{-b_{z} \mb{j} + b_{y} \mb{k}}{B(B - b_x)}
  \end{aligned}
\end{equation*}
Where we have used the fact that $\lim_{D \rightarrow \infty} \mb{d}/D
= -\mb{i}$.




\end{document}
